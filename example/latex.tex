\documentclass[polish,envcountsect,10pt]{beamer}
    \usepackage[T1]{fontenc}
    \usepackage{polski}
    \usepackage{babel}

    \usetheme{EastLansing}

    \title{\S\texorpdfstring{$2$}{2}. System składu tekstu {\fontfamily{lmr}\selectfont\mdseries\LaTeX}}
    \author{Robert Janczewski}
    \date{Gdańsk, \texorpdfstring{$2025$}{2025}}

\begin{document}

\frame{\titlepage}

\begin{frame}{Cechy systemu}
    \hspace*{0pt}{\fontfamily{lmr}\selectfont\mdseries\LaTeX} to język opisu dokumentów tekstowych, umożliwiający tworzenie wysokiej jakości dokumentów zawierających tekst, formuły matematyczne i
    grafikę.
    \medskip

    Opisując dokument przy pomocy {\fontfamily{lmr}\selectfont\mdseries\LaTeX}-a określamy jedynie jego zawartość i logiczną strukturę, a ustalenie szczegółów wyglądu zostawiamy
    {\fontfamily{lmr}\selectfont\mdseries\LaTeX}-owi.
    \medskip

    {\fontfamily{lmr}\selectfont\mdseries\LaTeX} jest de facto zbiorem makr dla systemu {\fontfamily{lmr}\selectfont\mdseries\TeX}, który zajmuje się właściwym składem dokumentu.
    \medskip

    {\fontfamily{lmr}\selectfont\mdseries\TeX} realizuje te elementy składu, które można zautomatyzować: podział tekstu na strony, układanie tekstu i grafik na stronie, numerowanie i odwołania do
    numerów, tworzenie spisów treści, bibliografii, indeksu itd.
\end{frame}

\begin{frame}{Narzędzia}
    Aby skorzystać z {\fontfamily{lmr}\selectfont\mdseries\LaTeX}-a, niezbędny jest edytor tekstu i jedna z dystrybucji {\fontfamily{lmr}\selectfont\mdseries\LaTeX}-a.
    \medskip

    Lista edytorów wspierających pracę z dokumentami {\fontfamily{lmr}\selectfont\mdseries\LaTeX}-owymi jest długa, podobnie jak lista dystrybucji {\fontfamily{lmr}\selectfont\mdseries\LaTeX}-a.
    \medskip

    Dla przykładu, w systemie Windows działa MikTeX (\url{https://miktex.org/}) i TeXnicCenter (\url{http://www.texniccenter.org/}), a pod Linuksem TeXLive (\url{https://www.tug.org/texlive/}) i
    kwrite (\url{https://www.kde.org/applications/utilities/kwrite/}).
    \medskip

    Istnieją także rozwiązania nie wymagające instalacji, ale wymagające dostępu do internetu, np.\ serwis \url{https://www.overleaf.com/}.
\end{frame}

\begin{frame}{Praca z {\fontfamily{lmr}\selectfont\mdseries\LaTeX}-em}
    Pierwszym etapem pracy z {\fontfamily{lmr}\selectfont\mdseries\LaTeX}-em jest przygotowanie pliku źródłowego, po nim następuje kompilacja (jedno- lub wielokrotna), dająca w wyniku docelowy
    dokument i rozmaite pliki pomocnicze.
    \medskip

    {\fontfamily{lmr}\selectfont\mdseries\LaTeX} korzysta m.in.\ z plików o następujących rozszerzeniach:
    \begin{enumerate}
        \item tex; plik źródłowy;
        \item cls; pliki źródłowe klas dokumentów;
        \item sty; pliki źródłowe pakietów makr;
        \item dvi/pdf; docelowe dokumenty;
        \item toc/lof/lot; spis treści, rysunków i tabel;
        \item aux; plik wykorzystywany przy wielokrotnej kompilacji;
        \item log; zapis przebiegu kompilacji.
    \end{enumerate}
\end{frame}

\begin{frame}{Praca z {\fontfamily{lmr}\selectfont\mdseries\LaTeX}-em}
    Efektem kompilacji jest także wyświetlany na ekranie raport, zawierający informacje o błędach składni, ostrzeżeniach i błędach składu.
    \medskip

    Ostrzeżenia pojawiają się wtedy, gdy pewne czcionki są niedostępne, gdy korzystamy z przestarzałych pakietów, gdy dokonano pewnych (niekoniecznie pożądanych przez autora) korekt w dokumencie itd.
    \medskip

    Błędy składu pojawiają się wtedy, gdy algorytmom stosowanym przez {\fontfamily{lmr}\selectfont\mdseries\TeX}-a nie udało się tak ułożyć tekstu, by dobrze wyglądał.
    \medskip

    Dostajemy wówczas najlepiej według {\fontfamily{lmr}\selectfont\mdseries\TeX}-a ułożony tekst, ale musimy go zmodyfikować, jeśli chcemy by był dobrze złożony.
    \medskip

    Dobrym zwyczajem jest takie przygotowanie finalnej wersji dokumentu, żeby przy jego kompilacji nie było ostrzeżeń ani błędów składu.
\end{frame}

\begin{frame}{Schemat pliku źródłowego}
    Typowy plik *.tex składa się z następujących fragmentów:
    \begin{enumerate}
        \item preambuły; preambuła musi określać klasę dokumentu i używane przez dokument pakiety, można tam także umieścić rozmaite polecenia, np.\ definiujące nowe makra i zmieniające wygląd
            dokumentu;
        \item dokumentu właściwego; zawiera on treść dokumentu docelowego oraz polecenia określające logiczną strukturę dokumentu.
    \end{enumerate}
    To, co znajduje się w preambule wpływa na wygląd dokumentu, ale nie generuje żadnej treści---ta musi znajdować się w dokumencie właściwym.
    \medskip

    Wszystko, co znajduje się po dokumencie właściwym jest ignorowane.
\end{frame}

\begin{frame}[fragile]{Przykład}
    Poniżej znajduje się fragment preambuły i prawie pusta właściwa część dokumentu.
    \begin{verbatim}
\documentclass[polish,10pt]{beamer}
    \usepackage[T1]{fontenc}
    \usepackage{polski}
    \usepackage{babel}
    \newcommand{\highlight}[1]{\alert{\emph{#1}}}
\begin{document}
    To jest \highlight{T E S T}.
\end{document} \end{verbatim}
    Istnieje wiele gotowych klas dokumentów przygotowanych z myślą o różnych typach dokumentów, np.\ book i amsbook dla książek, article i amsart dla artykułów oraz beamer dla prezentacji.
\end{frame}

\begin{frame}{Tryby pracy {\fontfamily{lmr}\selectfont\mdseries\LaTeX}-a}
    W trakcie pracy nad dokumentem {\fontfamily{lmr}\selectfont\mdseries\LaTeX} może znajdować się w jednym z poniższych trybów:
    \begin{enumerate}
        \item matematycznym; w tym trybie przetwarzane są formuły matematyczne, obowiązują wówczas zupełnie inne reguły składu: spacje są nieznaczące, każdy znak ma przypisaną interpretację (zmienna,
        operator, relacja itd.), dostępne są rozmaite polecenia specjalne (greckie litery, specjalne symbole itd.);
        \item tekstowym pionowym; w tym trybie składane są akapity, które następnie są dzielone na strony;
        \item tekstowym poziomym; w tym trybie składany jest pojedynczy akapit, który jest dzielony na zdania, zdania na wyrazy a wyrazy na litery i ligatury.
    \end{enumerate}
    Tryb matematyczny trzeba włączyć ręcznie, pozostałe tryby uruchamiane są automatycznie.
\end{frame}

\begin{frame}[fragile]{Tryb tekstowy}
    Akapity rozdzielamy pustą linią lub poleceniem \verb+\par+. Podział na linie i strony następuje automatycznie, ale można go wymusić---służą do tego polecenia \verb+\newline+ i \verb+\newpage+.
    \medskip

    Dokument to zwykły tekst, w którym znajdują się polecenia (makra). Istnieją dwa typy poleceń:
    \begin{enumerate}
        \item otoczenia postaci \verb+\begin{nazwa}+ \ldots\ \verb+\end{nazwa}+; otoczenie oddziaływuje wyłącznie na znajdujący się wewnątrz niego tekst;
        \item zwykłe polecenia postaci \verb+\nazwa+; wywoływany przez polecenie efekt obowiązuje od tego momentu do końca tekstu.
    \end{enumerate}
    Istnieją także jednoznakowe polecenia specjalne, ich omówienie znajduje się dalej.
\end{frame}

\begin{frame}[fragile]{Tryb matematyczny}
    Formuły otaczamy znakiem \$, powodując włączenie trybu matematycznego. Ten tryb włączany jest także przez rozmaite polecenia takie jak:
    \begin{enumerate}
        \item \$\$ \ldots\ \$\$ lub \verb+\begin{displaymath}+ \ldots \verb+\end{displaymath}+ lub \verb+\begin{equation*}+ \ldots \verb+\end{equation*}+; formuła zostanie złożona w postaci
            wyśrodkowanej, w osobnym akapicie;
        \item \verb+\begin{equation}+ \ldots \verb+\end{equation}+; formule zostanie nadany numer i zostanie ona złożona w postaci wyśrodkowanej, w osobnym akapicie;
        \item \verb+\begin{eqnarray}+ \ldots \verb+\end{eqnarray}+; powstanie wielowierszowa formuła numerowana (z gwiazdką na końcu---nienumerowana).
    \end{enumerate}
    Tych poleceń nie można zagnieżdżać; \verb+\ensuremath+ wymusza użycie trybu matematycznego.
\end{frame}

\begin{frame}[fragile]{Przykłady}
    \begin{tabular}{r|l}
        Źródło & Dokument docelowy \\ \hline
        \verb+$\{x\leq n\colon 2|x\}$+ & $\{x\leq n\colon 2|x\}$ \\ \hline
        \verb+$f\colon X\to Y$+ & $f\colon X\to Y$ \\ \hline
        \begin{tabular}{l}
            \verb.$\mathbb{R}_+\ni x\mapsto. \\
            \verb.f(x)=\lg x\in\mathbb{R}$.
        \end{tabular} & $\mathbb{R}_+\ni x\mapsto f(x)=\lg x\in\mathbb{R}$ \\ \hline
        \verb+$G=(V$, $E)$ jest grafem+ & $G=(V$, $E)$ jest grafem \\ \hline
        \verb+$\Delta=\Theta(n)$+ & $\Delta=\Theta(n)$ \\ \hline
        \begin{tabular}{l}
            \verb+$A=[a_{ij}^2]_{i,j=1+ \\
            \verb+,\ldots,n}$+
        \end{tabular} & $A=[a_{ij}^2]_{i,j=1,\ldots,n}$ \\ \hline
        \verb+$\frac{1}{2}\sqrt{5}$+ & $\frac{1}{2}\sqrt{5}$ \\ \hline
        \verb+$a'$, $a''$+ & $a'$, $a''$
    \end{tabular}
\end{frame}

\begin{frame}[fragile]{Przykłady}
    \begin{tabular}{r|l}
        Źródło & Dokument docelowy \\ \hline
        \begin{tabular}{l}
            \verb+$a=\begin{cases}+ \\
            \verb+1\text{,}&\text{gdy }b=0\\+ \\
            \verb+2\text{,}&\text{gdy }b\ne 0+ \\
            \verb+\end{cases}$+
        \end{tabular} & $a=\begin{cases} 1\text{,} & \text{gdy }b=0\\ 2\text{,} & \text{gdy }b\neq 0\end{cases}$ \\ \hline
        \begin{tabular}{l}
            \verb+$\displaystyle\left(+ \\
            \verb+\frac{\sum_{i=1}^na_i}{n}+ \\
            \verb+\right)^n\geqslant+ \\
            \verb+\prod_{i=1}^na_i$+
        \end{tabular} & $\displaystyle\left(\frac{\sum_{i=1}^na_i}{n}\right)^n\geqslant\prod_{i=1}^na_i$
    \end{tabular}
\end{frame}

\begin{frame}[fragile]{Znaki specjalne}
    Niektóre znaki mają specjalne znaczenie:
    \begin{enumerate}
        \item \#; używane w makrach do oznaczania argumentów, np.\ \#$1$ to pierwszy argument makra;
        \item \%; komentarz, wszystko od tego znaku do końca linii jest ignorowane;
        \item \$; polecenie przejścia do trybu matematycznego;
        \item \&; separator pól w tabelach;
        \item $\sim$; niełamliwa spacja;
        \item \_ i \verb+^+; indeks dolny i górny (działa wyłącznie w trybie matematycznym);
        \item \textbackslash; znak rozpoczynający polecenia;
        \item $\{$ i $\}$; znaki otaczające parametry makr lub fragmenty tekstu;
        \item \hspace*{0pt} [ i ]; znaki otaczające opcjonalne parametry makr.
    \end{enumerate}
\end{frame}

\begin{frame}[fragile]{Znaki specjalne}
    Sposoby uzyskania w tekście znaków specjalnych przedstawia poniższa tabela.
    \begin{center}
        \begin{tabular}{r|l}
            Żródło & Dokument docelowy \\
            \verb+\#+ & \# \\ \hline
            \verb+\%+ & \% \\ \hline
            \verb+\$+ & \$ \\ \hline
            \verb+\&+ & \& \\ \hline
            \verb+$\sim$+ & $\sim$ \\ \hline
            \verb+\_+ & \_ \\ \hline
            \verb+\verb.^.+ & \verb+^+ \\ \hline
            \verb+\textbackslash+ & \textbackslash \\ \hline
            \verb+$\{$+ & $\{$ \\ \hline
            \verb+$\}$+ & $\}$
        \end{tabular}
    \end{center}
\end{frame}

\begin{frame}[fragile]{Listy}
    Do tworzenia list używane są następujące otoczenia:
    \begin{enumerate}
        \item \verb+\begin{itemize}+ \ldots\ \verb+\end{itemize}+; tworzy listę wypunktowaną;
        \item \verb+\begin{enumerate}+ \ldots\ \verb+\end{enumerate}+; tworzy listę numerowaną;
        \item \verb+\begin{description}+ \ldots\ \verb+\end{description}+; tworzy listę opisową;
    \end{enumerate}
    Element listy zaczyna się od polecenia \verb+\item+ (\verb+\item[nazwa elementu]+ w przypadku listy opisowej).
    \medskip

    Listy można zagnieżdżać, można także zmienić znaki używane w wypunktowaniu i styl numerowania. Pakiet \verb+enumitem+ daje duże możliwości dopasowania stylu list do potrzeb.
\end{frame}

\begin{frame}[fragile]{Tabele}
    Tabele tworzymy przy pomocy otoczenia \verb+\begin{tabular}{opis kolumn}+ \ldots\ \verb+\end{tabular}+.
    \medskip

    Opis kolumn może zawierać litery \verb+l+, \verb+r+, \verb+c+ i pionowe kreski. Litery określają sposób wyrównywania zawartości kolumn, a kreski informują, czy kolumny mają być separowane
    pionowymi liniami.
    \medskip

    Wiersze tabeli kończymy poleceniem \verb+\\+ lub \verb+\\ \hline+, jeśli po wierszu ma być linia pozioma. Kolumny separujemy znakiem \&.
    \medskip

    Tabele są z zasady jednostronicowe. Jeśli chcemy korzystać z tabel automatycznie dzielonych między strony, należy skorzystać z odpowiedniego pakietu, np.\ \verb+longtable+.
\end{frame}

\begin{frame}[fragile]{Czcionki}
    Do zmiany rozmiaru czcionki można użyć wbudowanych poleceń, takich jak \verb+\tiny+, \verb+\normalfont+, \verb+\large+, \verb+\Large+, \verb+\LARGE+, \verb+\huge+ i \verb+\Huge+.
    \medskip

    Zmianę stylu czcionki uzyskujemy następującymi poleceniami:
    \begin{enumerate}
        \item \verb+\textbf{}+ lub \verb+\bfseries+; pogrubienie;
        \item \verb+\textmd{}+ lub \verb+\mdseries+; pismo normalnej grubości;
        \item \verb+\textrm{}+ lub \verb+\upshape+; pismo proste;
        \item \verb+\textit{}+ lub \verb+\itshape+; italik;
        \item \verb+\textsl{}+ lub \verb+\slshape+; pismo pochyłe;
        \item \verb+\textsc{}+ lub \verb+\scshape+; kapitaliki.
    \end{enumerate}
\end{frame}

\begin{frame}[fragile]{Grafika}
    Grafikę można dołączać z zewnętrznych plików, służy do tego polecenie \verb+\includegraphics+. Można ją także tworzyć przy pomocy pakietu \verb+tikz+.
    \medskip

    Tworzony obrazek umieszczamy wewnątrz otoczenia \verb+\begin{tikzpicture}+ \ldots\ \verb+\end{tikzpicture}+, a budujemy przy pomocy poleceń generujących proste kształty geometryczne:
    \verb+\filldraw+, \verb+\draw+ itd.
    \medskip

    Możliwe jest przy tym nadawanie punktom etykiet i stosowanie konstrukcji programistycznych typu pętle, instrukcje warunkowe i zmienne.
\end{frame}

\begin{frame}[fragile]{Algorytmy}
    Pakiet \verb+algorithm2e+ ułatwia zapis algorytmów w pseudokodzie. Algorytmy należy umieszczać wewnątrz otoczenia \verb+\begin{algorithm}+ \ldots\ \verb+\end{algorithm}+.
    \medskip

    Instrukcje występujące w algorytmach są generowane odpowiednimi poleceniami:
    \begin{enumerate}
        \item \verb+\Input{opis danych}+; opis wejścia;
        \item \verb+\Output{opis wyniku}+; opis wyjścia;
        \item \verb+\uIf{warunek}{instrukcje}+ i \verb+\Else{instrukcje}+; instrukcje warunkowe;
        \item \verb+\For{zakres pętli}{instrukcje}+; pętla.
    \end{enumerate}
    Powyższa lista nie jest kompletna, dostępnych jest dużo innych poleceń.
\end{frame}

\begin{frame}[fragile]{Organizacja dokumentu}
    Dostępne jednostki podziału dokumentu zależą od jego klasy.
    \medskip

    Książki oferują podział na część wstępną (\verb+\frontmatter+), główną (\verb+\mainmatter+) i tylną (\verb+\backmatter+). Oprócz tego dostępne są części (\verb+\part+), rozdziały
    (\verb+\chapter+), podrozdziały (\verb+\section+) i mniejsze jednostki: \verb+\subsection+, \verb+\subsubsection+ i \verb+\paragraph+.
    \medskip

    Mniejsze dokumenty oferują to samo, tyle, że nie ma w nich rozdziałów i części, a całość jest traktowana jak część główna.
\end{frame}

\begin{frame}[fragile]{Etykiety i automatyczna numeracja}
    Z każdym miejscem w tekście związany jest licznik, do którego można się dostać nadając mu etykietę.
    \medskip

    Etykietę tworzymy poleceniem \verb+\label{nazwa}+, a odczytujemy \verb+\ref{nazwa}+ (wartość) lub \verb+\pageref{nazwa}+ (numer strony, na którą trafiła).
    \medskip

    To, który licznik zostanie wykorzystany, jest ustalane automatycznie przez klasę dokumentu---wewnątrz rozdziału będzie to inny licznik niż wewnątrz twierdzenia czy rysunku.
\end{frame}

\begin{frame}[fragile]{Twierdzenia, dowody i definicje}
    Do tworzenia otoczeń typu twierdzenie służy pakiet \verb+amsthm+.
    \medskip

    Nowy typ twierdzenia tworzymy poleceniem \verb+\newtheorem{nazwa}{napis}[licznik]+. Nazwa określa nazwę tworzonego otoczenia, napis to napis pojawiający się na jego początku.
    \medskip

    Twierdzenie, do którego nie podajemy dowodu kończymy zawsze znakiem \verb+\qed+ (w dowodach pojawia się on automatycznie na końcu).
    \medskip

    Można zmienić styl twierdzenia przy pomocy polecenia \verb+\theoremstyle+. Dowody umieszczamy wewnątrz otoczenia \verb+\begin{proof}+ \ldots\ \verb+\end{proof}+.
\end{frame}

\begin{frame}[fragile]{Tworzenie makr}
    Nowe polecenia tworzymy pomocy polecenia \verb+\newcommand{nazwa}[liczba parametrów]{definicja}+.
    \medskip

    Nowe otoczenia tworzymy pomocy polecenia \verb+\newenvironment{nazwa}[liczba parametrów]+ \verb+{początek}{koniec}+. 
    \medskip

    ,,Początek'' jest wykonywany, kiedy w tekście pojawia się \verb+\begin{nazwa}+, a ,,koniec'', gdy pojawia się \verb+\end{nazwa}+.
    \medskip

    Można zmieniać istniejące polecenia, wystarczy zmienić \verb+new+ na \verb+renew+ w powyższych konstrukcjach.
    \medskip

    Jeśli nowe polecenie/otoczenie ma mieć parametry, to w części definiującej $i$-ty parametr widoczny jest jako \#$i$. Parametry otoczenia są widoczne wyłącznie w jego części początkowej.
\end{frame}

\begin{frame}[fragile]{Prezentacje}
    Prezentacje tworzymy przy pomocy klasy \verb+beamer+. Opis klasy beamer można znaleźć pod adresem \url{https://tug.ctan.org/macros/latex/contrib/beamer/doc/beameruserguide.pdf}.
    \medskip

    Prezentacja składa ze slajdów, przeznaczonych dla widowni oraz notatek, widocznych jedynie dla prelegenta. Notatki tworzymy poleceniem \verb+\note+.
    \medskip

    Slajdy mogą zawierać automatycznie tworzone panele nawigacyjne, informację o autorze i tytule, a także informację o tym, do którego rozdziału i podrozdziału należą.
    \medskip

    Treść slajdów umieszczamy wewnątrz otoczenia \verb+\begin{frame}+ \ldots\ \verb+\end{frame}+. O stylu slajdu decyduje wybrany wcześniej, zazwyczaj w preambule, styl dokumentu.
\end{frame}

\begin{frame}{Książki o {\fontfamily{lmr}\selectfont\mdseries\LaTeX}-u}
    \begin{thebibliography}{9}
        \bibitem{Lamp} Lamport L.: ,,System opracowywania dokumentów {\fontfamily{lmr}\selectfont\mdseries\LaTeX}. Podręcznik i przewodnik użytkownika'', WNT, Warszawa $2004$.
        \bibitem{Oet} Oetiker T., Partl H., Hyna I., Schlegl E., Przechlewski T., Kubiak R., Gołdasz J., Serwin M.: ,,Nie za krótkie wprowadzenie do systemu
            {\fontfamily{lmr}\selectfont\mdseries\LaTeXe} albo {\fontfamily{lmr}\selectfont\mdseries\LaTeXe} w $156$ minut'',
            \url{https://ctan.gust.org.pl/tex-archive/info/lshort/polish/lshort-pl.pdf}
        \bibitem{Goo} Goossens M., Mittelbach F., Samarin A.: ,,The {\fontfamily{lmr}\selectfont\mdseries\LaTeX} Companion'', Addison-Wesley $1994$.
        \bibitem{Rah} Goossens M., Rahtz S., Mittelbach F.: ,,The {\fontfamily{lmr}\selectfont\mdseries\LaTeX} Graphics Companion. Illustrating documents with
            {\fontfamily{lmr}\selectfont\mdseries\TeX} and PostScript'', Addison-Wesley $1997$.
    \end{thebibliography}
\end{frame}

\end{document}
